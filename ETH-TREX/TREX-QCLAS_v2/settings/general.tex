% Font stuff
\usepackage[utf8]{inputenc}
\usepackage[main=english,ngerman]{babel}
\usepackage[style=american]{csquotes}
\usepackage{mathpazo} %Palatino
\linespread{1.05} %a bit bigger for Palatino
\usepackage[scaled]{helvet} % ss
\usepackage{courier} % tt
\usepackage[T1]{fontenc}
\usepackage{upquote} %single quotation marks
\usepackage[nointegrals]{wasysym} % for permille sign \permil
\usepackage{textgreek} % to write upright capital Delta \textDelta
\usepackage{upgreek}
\usepackage{pifont}% http://ctan.org/pkg/pifont, for special symbols

% Page layout

\usepackage[a4paper]{geometry}
\geometry{paperwidth = 216mm} %paper A4 + 6 mm cut = 216
\geometry{paperheight = 303mm} %paper A4 + 6 mm cut = 303
\geometry{bindingoffset=3mm} %binding offset
\geometry{textwidth=0.8\paperwidth} %textwidth
\geometry{textheight=0.8\paperheight} %textheight
\geometry{headsep=0.5cm} %separation between the text and the header
\geometry{twoside} %twosided text - print version
% \geometry{twoside=false} %onesided text - e-version
% \geometry{showframe} %to show frame for each object on the page
\usepackage{lscape} %creates landscape page in portrait document
%\usepackage{setspace} %provides features for advance line spacing
%\doublespacing
\raggedbottom %all pages will not have exactly the same height, but ‘almost the same height’


% to have the broader margin on the binding side of the page, for print version
\let\tmp\oddsidemargin
\let\oddsidemargin\evensidemargin
\let\evensidemargin\tmp
\reversemarginpar

% \usepackage[a5paper]{geometry}
% \geometry{paperwidth = 154mm} %paper A5 + 6 mm cut
% \geometry{paperheight = 216mm} %paper A5 + 6 mm cut
% \geometry{bindingoffset=5.5mm} %binding offset
% \geometry{textwidth=10.5cm} %textwidth
% \geometry{textheight=16.6cm} %textheight
% \geometry{headsep=0.5cm} %separation between the text and the header
% \geometry{twoside} %twosided text
% \usepackage{lscape} %creates landscape page in portrait document
% %\usepackage{setspace} %provides features for advance line spacing
% %\doublespacing
% \raggedbottom %all pages will not have exactly the same height, but ‘almost the same height’


% Page heading
\usepackage{fancyhdr}
\pagestyle{fancy}
\renewcommand{\chaptermark}[1]{\markboth{#1}{}}
\fancypagestyle{Chapter}{
	\fancyhf{}
	\fancyhead[LE]{Ch. \thechapter | \leftmark}
	\fancyfoot[C]{\thepage}	
}
\fancypagestyle{plain}{
	\renewcommand{\headrulewidth}{0pt}
	\fancyhf{}
	\fancyfoot[c]{\thepage}
}
\fancypagestyle{NoChapter}{
	\fancyhf{}
	\fancyhead[LE]{\leftmark}
	\fancyfoot[c]{\thepage}
}

\setlength{\headheight}{15pt} % it should be at least 15 pt

\usepackage{emptypage} %removes headers and footers on empty pages

%section titles
% \usepackage[compact]{titlesec}
\usepackage{titlesec}
\titleformat{\chapter}{}{}{0em}{\bf\LARGE} % to remove "Chapter XY" at the beginning of each chapter

% Math stuff
\usepackage{amsmath}
\usepackage{amssymb}
\usepackage{mathtools}
\usepackage{isomath}
\usepackage[version-1-compatibility,load=prefixed,detect-weight=true, detect-family=true]{siunitx}
\sisetup{quotient-mode = fraction}
\usepackage{bm} % to be able to use bold in math mode \mathbf{}
\usepackage{xfrac}

% Figures, tables, captions
\usepackage{booktabs}
\usepackage{tabularx}
\usepackage{ltablex}
\usepackage[font=small, labelfont=bf]{caption}
\usepackage{afterpage}
\usepackage{array}
\usepackage[table,svgnames]{xcolor}
\usepackage{threeparttable} %to have notes under a table
\usepackage{multirow}
\usepackage{arydshln} % to make a horizontal dashed line

% Chemistry
\usepackage[version=4]{mhchem}

% Hyphenation
\usepackage{hyphenat}
% \tolerance = 7000
% \hyphenpenalty=10000 % to supress hyphenation completely

\tolerance=1
\emergencystretch=\maxdimen
\hyphenpenalty=10000
\hbadness=10000

% Graphics
\usepackage{graphicx}
\usepackage{rotating}
\usepackage{tikz}
\usepackage{pgfplots}
\pgfplotsset{compat=newest}
\usetikzlibrary{shapes.geometric}

% Misc
\usepackage{xspace}
% \usepackage[hidelinks]{hyperref}

% PDF metadata for electronic version
\usepackage[pdftex,
            hidelinks,
            pdfauthor={Kristyna Kantnerova},
            pdftitle={Analysis of clumped isotopes in nitrous oxide: method development and first applications},
            pdfsubject={PhD thesis},
            pdfkeywords={clumped isotopes, nitrous oxide, laser spectroscopy}]{hyperref}
            
%  \usepackage[svgnames]{xcolor}
 %\definecolor{Teal}{RGB}{0,128,128}
%  \usepackage{hyperref}
 \hypersetup{colorlinks,breaklinks,
             urlcolor=Teal, % color of url links
             linkcolor=black, % color in ToC
             citecolor=Teal} % color of reference year
            
\usepackage[defaultlines = 3, all]{nowidow}
\usepackage{subcaption}
\usepackage{epstopdf}
\usepackage{relsize}
\usepackage[final]{pdfpages}
%\usepackage{showframe}
\usepackage{etoolbox}
\usepackage{verbatim}
\usepackage{lipsum} %generates lorem ipsum text
\usepackage{tocloft} % to add dots to ToC
\renewcommand{\cftchapdotsep}{\cftdotsep} 
\usepackage[noabbrev]{cleveref}
\addtolength{\skip\footins}{2pc plus 5pt} % to set a space between a footnote and a main text

% for the \intexttt command
\usepackage{xstring}
\newcommand{\intexttt}[1]{%
  \saveexpandmode\noexpandarg%
  \texttt{\StrSubstitute[0]{#1}{ }{\textnormal{ }}}%
  \restoreexpandmode%
}

%------- BIBLIOGRAPHY SETTINGS --------------------------------

\usepackage[
  style=nature,%
  %style=science, article-title=true,%
  natbib=true,%
  clearlang=true,%
  backend=biber, %sets which code is used for bibliography
  citestyle=authoryear
]{biblatex}

\ExecuteBibliographyOptions{%
  %--- Backend --- --- ---
  bibwarn=true, %
  bibencoding=auto, % (ascii, inputenc, <encoding>)
  %--- Sorting --- --- ---
  sorting=nyt,
  % other options: 
  % nty        Sort by name, title, year.
  % nyt        Sort by name, year, title.
  % nyvt       Sort by name, year, volume, title.
  % anyt       Sort by alphabetic label, name, year, title.
  % anyvt      Sort by alphabetic label, name, year, volume, title.
  % ynt        Sort by year, name, title.
  % ydnt       Sort by year (descending), name, title.
  % none       Do not sort at all. All entries are processed in citation order.
  % debug      Sort by entry key. This is intended for debugging only.
  %
  sortcase=true,
%   sortlos=los, % (bib, los) The sorting order of the list of shorthands
  sortcites=true, % do/do not sort citations according to bib	
  %--- Dates --- --- ---
  date=comp,  % (short, long, terse, comp, iso8601)
  %	origdate=
  %	eventdate=
  %	urldate=
  %	alldates=
  urldate=long, %
  datezeros=true, %
  dateabbrev=true, %
  %--- General Options --- --- ---
  maxnames=1,
  minnames=1,
  maxbibnames=100, % do not abbreviate names in bibliography
  % autocite= %(plain, inline, footnote, superscript) 
  autopunct=true,
  language=auto,
  autolang=none, % (none, hyphen, other, other*)
  block=none, % (none, space, par, nbpar, ragged)
  notetype=foot+end, % (foot+end, footonly, endonly)
  hyperref=true, % (true, false, auto)
  backref=false,
  backrefstyle=three, % (none, three, two, two+, three+, all+)
  backrefsetstyle=setonly, %
  indexing=false, % 
  % options:
  % true       Enable indexing globally.
  % false      Disable indexing globally.
  % cite       Enable indexing in citations only.
  % bib        Enable indexing in the bibliography only.
  refsection=none, % (part, chapter, section, subsection)
  refsegment=none, % (none, part, chapter, section, subsection)
  abbreviate=true, % (true, false)
  defernumbers=false, % 
  punctfont=false, % 
  arxiv=abs, % (ps, pdf, format)	
  %--- Style Options --- --- ---	
  % The following options are provided by the standard styles
  isbn=false,%
  url=true,%
  doi=true,%
  eprint=false,%
  uniquename=false
}%	

% Suppress all date fields except the year
\AtEveryBibitem{%
  \clearfield{day}%
  \clearfield{month}%
  \clearfield{endday}%
  \clearfield{endmonth}%
}

\DeclareRedundantLanguages{en,EN,English}{english}

% Use only the first page number in a given range
\DeclareFieldFormat{pages}{\mkfirstpage{#1}}

% to have et al. not in italics
\renewbibmacro*{name:andothers}{% Based on name:andothers from biblatex.def
  \ifboolexpr{
    test {\ifnumequal{\value{listcount}}{\value{liststop}}}
    and
    test \ifmorenames
  }
    {\ifnumgreater{\value{liststop}}{1}
       {\finalandcomma}
       {}%
     \andothersdelim\bibstring{andothers}}
     % \andothersdelim\bibstring[\emph]{andothers}} for italics
    {}}
\newcounter{dummy}

%------- DOCUMENT SETTINGS ------------------------------------

% paragraphs
\setlength{\parskip}{10pt plus 2pt minus 2pt} %set space between paragraphs
\setlength\parindent{0pt} %no indent
%\setlength{\intextsep}{12pt plus 2pt minus 2pt} %set space between text and flows
%\setlength{\textfloatsep}{12pt plus 2pt minus 2pt} %set distance between floats and text
\AtBeginEnvironment{equation}{\vspace{-5pt plus 2pt minus 2pt}} %no paragraph space in front of the equation
\AtEndEnvironment{equation}{\vspace{-5pt plus 2pt minus 2pt}} %no paragraph space after the equation

% section titles
\titlespacing*{\section}{0pt}{12pt}{0pt}
\titlespacing*{\subsection}{0pt}{12pt}{0pt}

% Math stuff
\DeclareMathOperator\erfc{erfc}


% set SI units
\sisetup{
  separate-uncertainty,
  repeatunits=false,
  detect-family,
  unit-mode=text,
}
\DeclareSIUnit\au{arb. units}
\DeclareSIUnit\Torr{Torr}


% write this symbol: @
\newcommand{\at}{\makeatletter @\makeatother \xspace}

%------- ENVIRONMENT SETTING ------------------------------------

% Abstract 
\newenvironment{abstracts} {
	\begin{center}
    {\Large \bfseries  Abstract}
    \end{center}
}

% Zusammenfassung 
\newenvironment{zusammenfassung} {
	\begin{center}
    {\Large \bfseries  Zusammenfassung}
    \end{center}
}

% Dedication
\newenvironment{dedication} {
	\pagestyle{empty}
    \begin{center}
    \vspace*{1.5cm}
    {\LARGE }
    \end{center}
    \vspace{0.5cm}
    \begin{quote} \begin{center}}
    {\end{center} \end{quote}
}

% Acknowledgement
\newenvironment{acknowledgement} {
	\begin{center}
	{\Large \bfseries Acknowledgements}
	\end{center}
}

% Curriculum Vitae
\newenvironment{cv} {
  {\LARGE \bfseries  Curriculum Vitae}
}

% Absolutely, definitely, preventing page break
\newenvironment{absolutelynopagebreak}
  {\par\nobreak\vfil\penalty0\vfilneg
   \vtop\bgroup}
  {\par\xdef\tpd{\the\prevdepth}\egroup
   \prevdepth=\tpd}

% new column type for chapter stat_therm.tex
\newcolumntype{x}[1]{>{\centering\let\newline\\\arraybackslash\hspace{0pt}}m{#1}}
%\newcolumntype{x}[1]{>{\centering\arraybackslash}m{#1}}


% having the links in ToC clickable with the package hyperref
\renewcommand{\theHchapter}{\arabic{chapter}\thechapter}

% be able to change the ToC depth mid-document
\newcommand{\changelocaltocdepth}[1]{%
  \addtocontents{toc}{\protect\setcounter{tocdepth}{#1}}%
  \setcounter{tocdepth}{#1}%
}

\setcounter{tocdepth}{2}

% having list of abbreviations sorted alphabetically
\usepackage{datatool}% http://ctan.org/pkg/datatool
\newcommand{\sortitem}[1]{%
  \DTLnewrow{list}% Create a new entry
  \DTLnewdbentry{list}{description}{#1}% Add entry as description
}
\newenvironment{sortedlist}{%
  \DTLifdbexists{list}{\DTLcleardb{list}}{\DTLnewdb{list}}% Create new/discard old list
}{%
  \DTLsort{description}{list}% Sort list
  \begin{itemize}%
    \DTLforeach*{list}{\theDesc=description}{%
      \item \theDesc}% Print each item
  \end{itemize}%
}

